\documentclass[12pt]{article}
\title{ECE M16 Homework 1}
\usepackage{subcaption}
\author{Lawrence Liu}
\usepackage{graphicx}
\usepackage{amsmath}
\usepackage{placeins}
\newcommand{\Laplace}{\mathscr{L}}
\setlength{\parskip}{\baselineskip}%
\setlength{\parindent}{0pt}%
\usepackage{xcolor}
\usepackage{listings}
\definecolor{backcolour}{rgb}{0.95,0.95,0.92}
\usepackage{amssymb}
\lstdefinestyle{mystyle}{
    backgroundcolor=\color{backcolour}}
\lstset{style=mystyle}

\begin{document}
\maketitle
\section*{Problem 1}
Since there are 26 letters in the English Alphabet, we would need $\lceil \log_2 (26)\rceil=5$ bits to represent this signal.
Therefore we could create a way of encoding the English Alphabet as 5 bits with gray encoding. ie we would have the following table
\begin{center}
    \begin{tabular}{|c|c|}
        a & 00000 \\
        \hline
        b & 00001 \\
        \hline
        c & 00011 \\
        \hline
        d & 00010 \\
        \hline
        e & 00110 \\
        \hline
        f & 00111 \\
        \hline
        g & 00101 \\
        \hline
        h & 00100 \\
        \hline
        i & 01100 \\
        \hline
        j & 01101 \\
        \hline
        k & 01111 \\
        \hline
        l & 01110 \\
        \hline
        m & 01010 \\
        \hline
    \end{tabular}
    \quad
    \begin{tabular}{|c|c|}
        n & 01011 \\
        \hline
        o & 01001 \\
        \hline
        p & 01000 \\
        \hline
        q & 11000 \\
        \hline
        r & 11001 \\
        \hline
        s & 11011 \\
        \hline
        t & 11010 \\
        \hline
        u & 11110 \\
        \hline
        v & 11111 \\
        \hline
        w & 11101 \\
        \hline
        x & 11100 \\
        \hline
        y & 10100 \\
        \hline
        z & 10101 \\
        \hline
    \end{tabular}
\end{center}
\section*{Problem 2}
\subsection*{(a)}
The equation for the circuit is 
$$f(a,b,c)=((a\wedge \bar{b})\vee\bar{c})\wedge\overline{((c\vee\bar{a})\wedge b)}$$
Expanding it we get
\begin{align*}
    f(a,b,c)&=((a\wedge \bar{b})\vee\bar{c})\wedge(\overline{(c\vee\bar{a})}\vee \overline{b})\\
    &=((a\wedge \bar{b})\vee\bar{c})\wedge((\overline{c}\wedge a)\vee \overline{b})\\
    &=((a\wedge \bar{b})\wedge((\overline{c}\wedge a)\vee \overline{b}))\vee(\bar{c}\wedge((\overline{c}\wedge a)\vee \overline{b}))\\
    &=\boxed{(a\wedge \bar{b}\wedge\overline{c})\vee(a\wedge \bar{b})\vee(\overline{c}\wedge a)\vee (\overline{c}\wedge\overline{b})}
\end{align*}
\subsection*{(b)}
\begin{align*}
    f(a,b,c)&=(a\wedge \bar{b}\wedge\overline{c})\vee(a\wedge \bar{b})\vee(\overline{c}\wedge a)\vee (\overline{c}\wedge\overline{b})\\
    &=\boxed{(a\wedge \bar{b})\vee(\overline{c}\wedge a)\vee (\overline{c}\wedge\overline{b})}\\
\end{align*}
\section*{Problem 3}
\subsection*{(a)}
we have $a.\bar{a}=0$, therefore we have
\begin{align*}
    a+0&=a\\
    a+(a.\bar{a})&=a\\
    (a+a).(a+\bar{a})&=a\\
    (a+a).1&=a\\
    a+a&=a
\end{align*}
Likewise we have
\begin{align*}
    a.1&=a\\
    a.(a+\bar{a})&=a\\
    a.a+a.\bar{a}&=a\\
    a.a+0&=a\\
    a.a&=a
\end{align*}
\subsection*{(b)}
From the Boolean Algebra postulates we have: 
$$1.\bar{1}=0$$
Therefore we must have that $\bar{1}=0$
\subsection*{(c)}

Let us prove that $\bar{a}$ is unique through contradiction, ie: for a given $a$, there are
two variables, $\bar{a_1}$ and $\bar{a_2}$, that satisfy the complement law:
$$\bar{a_1}.a=\bar{a_2}.a=0$$
$$\bar{a_1}+a=\bar{a_2}+a=1$$
We also have

$$\bar{a_1}.a.\bar{a_2}=(\bar{a_1}.a).\bar{a_2}=\bar{a_2}$$
And
$$\bar{a_1}.a.\bar{a_2}=\bar{a_1}.(a.\bar{a_2})=\bar{a_1}$$

Therefore $\bar{a_1}=\bar{a_2}$, and thus $\bar{a}$ must be unique for a given $a$.
\section*{Problem 4}
\subsection*{(a)}
\begin{center}
    \begin{tabular}{ |c|c|c||c| }
        p & q & r  & f\\
        \hline
        1 & 1 & 1 & 1\\
        \hline
        1 & 1 & 0 & 1\\
        \hline
        1 & 0 & 1 & 0\\
        \hline
        1 & 0 & 0 & 0\\
        \hline
        0 & 1 & 1 & 1\\
        \hline
        0 & 1 & 0 & 0\\
        \hline
        0 & 0 & 1 & 1\\
        \hline
        0 & 0 & 0 & 0\\
        \hline

    \end{tabular}
\end{center}
% \subsection*{(b)}
% \includegraphics[scale=0.1]{Kmap.jpg}
\subsection*{(c)}
\includegraphics[scale=0.1]{KmapPrime.jpg}\\
Therefore the prime implicants are 
$$\boxed{\bar{p}.r}$$
$$\boxed{q.r}$$
$$\boxed{p.q}$$
\subsection*{(d)}
\includegraphics[scale=0.1]{KmapFinal.jpg}\\
The essential prime implicants are
$$\bar{p}.r$$
and 
$$p.q$$
Therefore the boolean expression for the function is 
$$f(p,q,r)=\boxed{(\bar{p}\wedge r)\vee(p\wedge q)=(\bar{p}.r)+(p.q)}$$
\section*{Problem 5}
\begin{center}
    \begin{tabular}{ |c|c|c||c|c| }
        x & y & z & $\overline{x+y+z}$ & $\overline{x}.\bar{y}.\bar{z}$\\
        \hline
        1 & 1 & 1 & 0 & 0\\
        \hline
        1 & 1 & 0 & 0 & 0\\
        \hline
        1 & 0 & 1 & 0 & 0\\
        \hline
        1 & 0 & 0 & 0 & 0\\
        \hline
        0 & 1 & 1 & 0 & 0\\
        \hline
        0 & 1 & 0 & 0 & 0\\
        \hline
        0 & 0 & 1 & 0 & 0\\
        \hline
        0 & 0 & 0 & 1 & 1\\
        \hline
    \end{tabular}
\end{center}

\begin{center}
    \begin{tabular}{ |c|c|c||c|c| }
        x & y & z & $\overline{x.y.z}$ & $\overline{x}+\bar{y}+\bar{z}$\\
        \hline
        1 & 1 & 1 & 0 & 0\\
        \hline
        1 & 1 & 0 & 1 & 1\\
        \hline
        1 & 0 & 1 & 1 & 1\\
        \hline
        1 & 0 & 0 & 1 & 1\\
        \hline
        0 & 1 & 1 & 1 & 1\\
        \hline
        0 & 1 & 0 & 1 & 1\\
        \hline
        0 & 0 & 1 & 1 & 1\\
        \hline
        0 & 0 & 0 & 1 & 1\\
        \hline
    \end{tabular}
\end{center}
\section*{Problem 6}
\begin{align*}
    (y.\bar{z}+\bar{x}.w).(x.\bar{y}+z.\bar{w})&=y.\bar{z}.(x.\bar{y}+z.\bar{w})+\bar{x}.w.(x.\bar{y}+z.\bar{w})\\
    &=\bar{z}.x.y.\bar{y}+y.\bar{w}.z.\bar{z}+w.\bar{y}.\bar{x}.x+z.\bar{x}.w.\bar{w}
    &=\boxed{0}
\end{align*}
since $x.\bar{x}=y.\bar{y}=w.\bar{w}=z.\bar{z}=0$
\begin{align*}
    (x.y)+(x.(w.z+w.\bar{z}))&=\boxed{(x.y)+(x.w)}
\end{align*}
\end{document}