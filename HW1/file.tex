\documentclass[12pt]{article}
\title{ECE M16 Homework 1}
\usepackage{subcaption}
\author{Lawrence Liu}
\usepackage{graphicx}
\usepackage{amsmath}
\usepackage{placeins}
\newcommand{\Laplace}{\mathscr{L}}
\setlength{\parskip}{\baselineskip}%
\setlength{\parindent}{0pt}%
\usepackage{xcolor}
\usepackage{listings}
\definecolor{backcolour}{rgb}{0.95,0.95,0.92}
\usepackage{amssymb}
\lstdefinestyle{mystyle}{
    backgroundcolor=\color{backcolour}}
\lstset{style=mystyle}

\begin{document}
\maketitle
\section*{Problem 1}
Since there are 26 letters in the English Alphabet, we would need $\lceil \log_2 (26)\rceil=5$ bits to represent this signal.
Therefore we could create a way of encoding the English Alphabet as 5 bits with each letter being encoded as 1+ the encoding of the previous letter, for instance 
A=00000 and B=00001, etc.
\section*{Problem 2}
\subsection*{(a)}
The equation for the circuit is 
$$f(a,b,c)=((a\vee \bar{b})\wedge\bar{c})\vee\overline{((c\wedge\bar{a})\vee b)}$$
Expanding it we get
\begin{align*}
    f(a,b,c)&=((a\wedge\bar{c})\vee (\bar{b}\wedge\bar{c}))
                \vee\overline{((c\wedge\bar{a})\vee b)}\\
    &=((a\wedge\bar{c})\vee (\bar{b}\wedge\bar{c}))
    \vee\overline{((c\vee b)\wedge(\bar{a}\vee b))}\\
    &=((a\wedge\bar{c})\vee (\bar{b}\wedge\bar{c}))
    \vee(\overline{(c\vee b)}\vee\overline{(\bar{a}\vee b)})\\
    &=(a\wedge\bar{c})\vee (\bar{b}\wedge\bar{c})
    \vee(\bar{c}\wedge \bar{b})\vee(a\wedge\bar{b})\\
    &=\boxed{(a\wedge\bar{c})
    \vee(\bar{c}\wedge \bar{b})\vee(a\wedge\bar{b})}
\end{align*}
\subsection*{(b)}
$$\boxed{(a\wedge\bar{c})
\vee(\bar{c}\wedge \bar{b})\vee(a\wedge\bar{b})}$$

\end{document}