\documentclass[12pt]{article}
\title{ECE M16 Final}
\usepackage{subcaption}
\author{Lawrence Liu}
\usepackage{graphicx}
\usepackage[english,shorthands=off]{babel}        % shorhands=off is required for babel french in combination with tikz karnaugh....
\usepackage[utf8x]{inputenc}
\usepackage[T1]{fontenc}
\usepackage{amsmath}
\usepackage{geometry}
\geometry{verbose,a4paper, tmargin=3.5cm,bmargin=3.5cm,lmargin=2.5cm,rmargin=2.5cm,headsep=1cm,footskip=1.5cm}
\usepackage{colortbl}
\usepackage[dvipsnames]{xcolor}
\usepackage{tikz -timing}
\usepackage{tikz}
\usepackage{listings}
\usetikzlibrary{karnaugh}

\definecolor{LogisimKMapColor0}{RGB}{128,0,0}
\definecolor{LogisimKMapColor1}{RGB}{230,25,75}
\definecolor{LogisimKMapColor2}{RGB}{250,190,190}
\definecolor{LogisimKMapColor3}{RGB}{170,110,40}
\definecolor{LogisimKMapColor4}{RGB}{245,130,48}
\definecolor{LogisimKMapColor5}{RGB}{255,215,180}
\definecolor{LogisimKMapColor6}{RGB}{128,128,0}
\definecolor{LogisimKMapColor7}{RGB}{255,255,25}
\definecolor{LogisimKMapColor8}{RGB}{210,245,60}
\definecolor{LogisimKMapColor9}{RGB}{0,0,128}
\definecolor{LogisimKMapColor10}{RGB}{145,30,180}
\definecolor{LogisimKMapColor11}{RGB}{60,180,175}
\definecolor{LogisimKMapColor12}{RGB}{0,130,203}
\definecolor{LogisimKMapColor13}{RGB}{230,190,255}
\definecolor{LogisimKMapColor14}{RGB}{170,255,195}
\definecolor{LogisimKMapColor15}{RGB}{240,50,230}


\definecolor{codegreen}{rgb}{0,0.6,0}
\definecolor{codegray}{rgb}{0.5,0.5,0.5}
\definecolor{codepurple}{rgb}{0.58,0,0.82}
\definecolor{backcolour}{rgb}{0.95,0.95,0.92}

\lstdefinestyle{mystyle}{
    backgroundcolor=\color{backcolour},   
    commentstyle=\color{codegreen},
    keywordstyle=\color{magenta},
    numberstyle=\tiny\color{codegray},
    stringstyle=\color{codepurple},
    basicstyle=\ttfamily\footnotesize,
    breakatwhitespace=false,         
    breaklines=true,                 
    captionpos=b,                    
    keepspaces=true,                 
    numbers=left,                    
    numbersep=5pt,                  
    showspaces=false,                
    showstringspaces=false,
    showtabs=false,                  
    tabsize=2
}

\lstset{style=mystyle}

\begin{document}
\maketitle
\section*{Problem 1}
\begin{center}
\begin{tabular}{c| c c c c}
    & 1 & 1 & 0 & 1\\
    \cline{2-5}
    1 & 11 & 00 & 00 & 10 \\
    & 1 & & &\\
    \cline{2-2}
    101 & 10 & 00 & &\\
    & 1 & 01 & &\\
    \cline{2-3}
    1100 & & 11 & 00 &\\
    & & 0 & &\\
    \cline{3-4}
    & & 11 & 00 & 10\\
    11001 & & 1 & 10 & 01\\
    \cline{3-5}\\
    & & 1 & 10 & 01
    
\end{tabular}
\end{center}
\section*{Problem 2}
Basing on the assumption that $c[1:0]=11$ corresponds with $o[1:0]=i[1:0]$
we have that 
\begin{align*}
    a[1]&=\overline{c[1]}.\overline{c[0]}.i[7]+\overline{c[1]}.c[0].i[5]+
        c[1].\overline{c[0]}.i[3]+c[1].c[0].i[1]\\
        &=\overline{c[1]}.(\overline{c[0]}.i[7]+c[0].i[5])+c[1].(\overline{c[0]}.i[3]+c[0].i[1])
\end{align*}
\begin{align*}
    a[0]&=\overline{c[1]}.\overline{c[0]}.i[6]+\overline{c[1]}.c[0].i[4]+
        c[1].\overline{c[0]}.i[2]+c[1].c[0].i[0]\\
        &=\overline{c[1]}.(\overline{c[0]}.i[6]+c[0].i[4])+c[1].(\overline{c[0]}.i[2]+c[0].i[0])
\end{align*}
Which results in a circuit like this\\
\includegraphics[scale=0.25]{fig1.png}
\section*{Problem 3}
I created the circuit, and it is shown above, and I tested it with the following python checker srcipt
\lstinputlisting[language=Python]{checker.py}
This script utilizes Logisim's command line ability. I had the files in the following format
\begin{verbatim}
ECEM16
 |- .gitignore
 |- Final
 | |- Logisim
 | | |- FinalQ3.circ
 | :
 | :
 | |- checker.py
 |- .gitignore
 |- logisim-evolution.jar
\end{verbatim}
\section*{Problem 4}
Let $c[1:0]$ be the desired output from our counter: we want the following timing diagram:
\\\includegraphics[scale=0.15]{fig4.jpg}\\
The first thing we notice is that $c[0]$ is just the clock input 
but with a period twice of clock, and $c[1]$ is just the clock input 
with a period of 4 times clock. We can make the output from a circuit 
toggle with each clock pulse (ie doubling its frequency), by connecting
$\overline{Q}$ with $D$, in the following manner:\\
\includegraphics[scale=0.5]{fig5.png}\\
Likewise we can make the output from a circuit toggle with every other clock pulse 
(thereby multiplying its period by 4) by connecting two flip flops in the following manner\\
\includegraphics[scale=0.5]{fig6.png}\\
Therefore we have the following circuit:\\
\includegraphics[scale=0.25]{fig7.png}
\section*{Problem 5}
I created the circuit, and it is shown above, and I tested 
by comparing its output vs that of a logisim counter, its output is denoted
as test in the chronograph below.\\
\includegraphics[scale=0.2]{fig8.png}
As one can see, the chronograph outputs of my counter vs logisim's counter is the same.
\section*{Problem 6}
We can break it up into two circuits, one for the GO signal with inputs of REQ and DONE, and one for the ACK signal with an input of DONE
For the ACK circuit we have the following transition table:
\begin{center}
    
    \begin{tabular}{|c|c|c|}
        Current State DONE,ACK & input (DONE) & Next State \\
        \hline
        0,0 & 0 & 0\\
        \hline
        0,0 & 1 & 1\\
        \hline
        0,1 & 0 & 1\\
        \hline
        0,1 & 1 & 0\\
        \hline
        1,1 & 0 & 0\\
        \hline
        1,1 & 1 & 1\\
        \hline
        1,0 & 0 & 1\\
        \hline
        1,0 & 1 & 0\\
        \hline
    \end{tabular}
\end{center}
This can be implemented with a flip flop with the clock input as DONE and the
output $Q$ as ACK, and $\overline{Q}$ being fed into $D$. The transition table for\
the GO circuit is the following:
\begin{center}
    
    \begin{tabular}{|c|c|c|}
        Current State REQ,DONE,GO & input (REQ,DONE) & Next State (GO) \\
        \hline
        0,0,x & 0,x & 0\\
        \hline
        0,0,x & 1,x & 1\\
        \hline
        0,1,x & 0,x & 0\\
        \hline
        0,1,x & 1,x & 1\\
        \hline
        1,0,x & 0,x & 1\\
        \hline
        1,0,x & 1,x & 1\\
        \hline
        x,0,1 & x,0 & 1\\
        \hline
        x,0,1 & x,1 & 0\\
        \hline
        0,1 & 0 & 1\\
        \hline
        0,1 & 1 & 0\\
        \hline
        1,1 & 0 & 0\\
        \hline
        1,1 & 1 & 1\\
        \hline
        1,0 & 0 & 1\\
        \hline
        1,0 & 1 & 0\\
        \hline
    \end{tabular}
\end{center}
When constructing this transition tables, I made the following assumptions,
that immediately after DONE goes from 0 to 1, the GO signal must be dropped,
as it appears on the example timing diagram. 
\section*{Problem 7}
\section*{Problem 8}
Using the following flow chart\\
\includegraphics[scale=0.25]{fig9.jpg}\\
I created the circuit\\
\includegraphics[scale=0.25]{fig10.png}
\end{document}
